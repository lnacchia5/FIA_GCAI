\documentclass{article}

\usepackage{float}
\usepackage[italian]{babel}
\usepackage[letterpaper,top=3cm,bottom=3cm,left=3cm,right=3cm,marginparwidth=1.75cm]{geometry}
\usepackage{amsmath}
\usepackage{graphicx}
\usepackage[colorlinks=true, allcolors=black]{hyperref}




\title{Università degli studi di Salerno
        \\Fondamenti di Intelligenza Artificiale
        \\ GCPP : Graphic Card Price Predictor}
\author{autore: Luigi Nacchia\\matricola: 0512105854}
\date{}

\begin{document}

\begin{figure}
    \centering
    \includegraphics[width=0.95\linewidth]{logo_unisa.png}
\end{figure}

\maketitle

\newpage
\tableofcontents

\newpage
\section{Introduzione}
Negli ultimi anni, l'aumento esponenziale di mining farm e dello sviluppo modelli ML da addestrare ha portato ad un massiccio aumento della domanda di componenti hardware necessarie per tali attività, con conseguente aumento dei loro prezzi.

    \subsection{Obiettivi}
    Questo progetto di Fondamenti di Intelligenza Artificiale si propone di sviluppare un predittore del prezzo di lancio di nuove componenti, basato sulle caratteristiche ed i prezzi di lancio passati.
    \\
    Con prezzo di lancio s'intende il prezzo di vendita in dollari dei produttori originali delle schede grafiche(Nvidia, Amd e Intel) nel momento in cui vengono annunciate al pubblico.
    \\
    Gli obbiettivi principali del progetto includono:
    \begin{itemize}
        \item un'analisi dei dati delle schede grafiche già lanciate 
        \item l'identificazione delle feature più rilevanti
        \item l'implementazione di un modello di machine learning in grado di fare predizioni sul prezzo di lancio di nuove schede grafiche non ancora lanciate.
    \end{itemize}
    
    \subsection{Specifica PEAS}
        \begin{itemize}
        \item Performance (misura di prestazioni): precisione e accuratezza del modello
        \item Enviroment (ambiente): i vari datasets delle gpu già lanciate
        \item Actuators (azioni possibili): prevedere un prezzo di una nuova gpu
        \item Sensors (percezioni possibili): acquisizione di dati utili per la predizione
        \end{itemize}
    \subsection{Caratteristiche dell'ambiente}
        \begin{itemize}
        \item Completamente osservabile, in quanto si conoscono tutte le informazioni sulle gpu già lanciate.
        \item Deterministico, in quanto ogni predizione è influenzata dai dati su cui il modello viene addestrato.
        \item Episodico, in quanto ogni predizione non influenza le predizioni future. 
        \item Statico, in quanto l'ambiente(dataset) non cambia dopo una predizione.
        \item Continuo, in quanto non c'è un limite al numero di possibili predizioni
        \end{itemize}
    
    \subsection{Analisi del problema}
    GCPP mira a costruire un predittore di prezzi, si tratta quindi di un problema di apprendimento supervisionato.
    Le tecnologie che ho utilizzato per lo sviluppo del progetto sono:
        \begin{itemize}
        \item Python (SickitLearn, Pandas),
        \item Google Colab (notebook)
        \item GitHub (versioning)
        \item Overleaf (documentazione in \LaTeX)
        \end{itemize}

\newpage
\section{Data Understanding}
    \subsection{Acquisizione dei dati}
    L'acquisizione dei dati è il processo di raccolta, ed organizzazione dei dati necessari per andare a creare un modello di ML. 
    Con gli obiettivi chiari e definiti, sono andato alla ricerca di un dataset su Kaggle
    \subsection{Esplorazione dei dati}
    \subsection{Analisi della qualità dei dati}

\newpage
\section{Data Preparation}
    \subsection{Data Cleaning}
    \subsection{Feature Scaling}
    \subsection{Feature Selection}
    \subsection{Data Balancing}
    \subsection{Split del Dataset}

\newpage
\section{Data Modeling}
    \subsection{Scelta dell'algoritmo}
    \subsection{Addestramento}

\newpage
\section{Evaluation}

\newpage
\section{Deployment}


\end{document}
